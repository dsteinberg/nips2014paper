%% Define bracket commands (normal, square and curly).
\newcommand{\brac} [1]  {\ensuremath{\left({#1}\right)}}
\newcommand{\sbrac}[1]  {\ensuremath{\left[{#1}\right]}}
\newcommand{\cbrac}[1]  {\ensuremath{\left\{{#1}\right\}}}
\newcommand{\abrac}[1]  {\ensuremath{\left\langle{#1}\right\rangle}}

%% Symbols
\newcommand{\real}  [1] {\ensuremath{\mathbb{R}^{#1}}}
\newcommand{\ident} [1] {\ensuremath{\mathbf{I}_{#1}}}
\newcommand{\lstate}    {\ensuremath{\mathbf{x}}}
\newcommand{\lcov}      {\ensuremath{\mathbf{R}}}
\newcommand{\obs}       {\ensuremath{\mathbf{y}}}
\newcommand{\hyper}     {\ensuremath{\boldsymbol{\theta}}}
\newcommand{\prmean}    {\ensuremath{\boldsymbol{\mu}}}
\newcommand{\prcov}     {\ensuremath{\boldsymbol{\Sigma}}}
\newcommand{\pomean}    {\ensuremath{\mathbf{m}}}
\newcommand{\pomeanp}   {\ensuremath{\mathbf{m}_\text{-}}}
\newcommand{\pocov}     {\ensuremath{\mathbf{C}}}
\newcommand{\pocovp}    {\ensuremath{\mathbf{C}_\text{-}}}
\newcommand{\xcov}      {\ensuremath{\boldsymbol\Sigma_{\obs\pomean}}}
\newcommand{\Sobs}      {\ensuremath{\mathcal{Y}}}
\newcommand{\Sfunc}     {\ensuremath{\mathcal{M}}}
\newcommand{\scoef}     {\ensuremath{\kappa}}
\newcommand{\Sw}        {\ensuremath{w}}
\newcommand{\Kgain}     {\ensuremath{\mathbf{H}}}
\newcommand{\Linmat}    {\ensuremath{\mathbf{A}}}
\newcommand{\intcpt}    {\ensuremath{\mathbf{b}}}
\newcommand{\Fengy}     {\ensuremath{\mathcal{F}}}
\newcommand{\step}      {\ensuremath{\alpha}}
\newcommand{\jacob}[1]  {\ensuremath{\mathbf{J}_{#1}}}

% Augmented systems
\newcommand{\augobs}    {\ensuremath{\mathbf{z}}}
\newcommand{\augcov}    {\ensuremath{\mathbf{S}}}
\newcommand{\augLinmat} {\ensuremath{\mathbf{B}}}
\newcommand{\augintcpt} {\ensuremath{\mathbf{c}}}

% Gaussian Process
\newcommand{\obss}      {\ensuremath{y}}
\newcommand{\lstates}   {\ensuremath{x}}
\newcommand{\Lins}      {\ensuremath{a}}
\newcommand{\Linvec}    {\ensuremath{\mathbf{a}}}
\newcommand{\intcpts}   {\ensuremath{b}}
\newcommand{\inobs}     {\ensuremath{\mathbf{s}}}
\newcommand{\kernl}     {\ensuremath{k}}
\newcommand{\Kernl}     {\ensuremath{\mathbf{k}}}
\newcommand{\KERNL}     {\ensuremath{\mathbf{K}}}
\newcommand{\lvar}      {\ensuremath{\sigma^2}}
\newcommand{\lstd}      {\ensuremath{\sigma}}
\newcommand{\Lvar}      {\ensuremath{\boldsymbol\Lambda}}
\newcommand{\pomeans}   {\ensuremath{m}}
\newcommand{\pocovs}    {\ensuremath{C}}
\newcommand{\xcovs}     {\ensuremath{\Sigma}}
\newcommand{\khyper}    {\ensuremath{\theta}}
\newcommand{\khypers}    {\ensuremath{\boldsymbol\theta}}


%% Operations
\newcommand{\transpose}  {\ensuremath{^{\!\top}}}
\newcommand{\inv}        {\ensuremath{^{\text{-}1}}}
\newcommand{\deter}[1]   {\ensuremath{\left|{#1}\right|}}
\newcommand{\trace}[1]   {\ensuremath{\text{tr}\!\brac{#1}}}
\newcommand{\diag}[1]    {\ensuremath{\text{diag}\!\brac{#1}}}
\newcommand{\expec}[2]   {\ensuremath{\abrac{#2}_{#1}}}
\newcommand{\expece}[2]  {\ensuremath{\mathbb{E}_{#1}\!\sbrac{#2}}}
\newcommand{\evar} [2]   {\ensuremath{\mathbb{V}_{#1}\!\sbrac{#2}}}
\newcommand{\KL}[2]      {\ensuremath{\text{KL}\!\sbrac{{#1}\!\parallel\!{#2}}}}
\newcommand{\entropy}[1] {\ensuremath{\mathbb{H}\sbrac{#1}}}
\newcommand{\test}       {\ensuremath{^{*}}}
\newcommand{\testT}      {\ensuremath{^{*\top}\!}}
\newcommand{\ttest}      {\ensuremath{^{**}}}
\newcommand{\lnorm}[2]   {\ensuremath{\left\|{#2}\right\|_{{#1}}}}


%% Functions, PDFs etc
\newcommand{\nonlin}[1] {\ensuremath{f\!\brac{{#1}}}}
\newcommand{\augnonlin}[1] {\ensuremath{g\!\brac{{#1}}}}
\newcommand{\prob}  [1] {\ensuremath{p\!\brac{#1}}}
\newcommand{\probC} [2] {\ensuremath{p\!\left({#1}\middle\vert{#2}\right)}}
\newcommand{\qrob}  [1] {\ensuremath{q\!\brac{#1}}}
\newcommand{\qrobC} [2] {\ensuremath{q\!\left({#1}\middle\vert{#2}\right)}}
\newcommand{\gaus}  [1] {\ensuremath{\mathcal{N}\!\brac{#1}}}
\newcommand{\gausC} [2] {\ensuremath{\mathcal{N}\!\left({#1}\middle\vert{#2}\right)}}
\newcommand{\kfunc} [2] {\ensuremath{\kernl\!\brac{{#1}, {#2}}}}
\newcommand{\expon} [2] {\ensuremath{{#1}\!\times\!10^{#2}}}

%% Operators
\DeclareMathOperator*{\argmax}{\operatorname*{argmax}}
\DeclareMathOperator*{\argmin}{\operatorname*{argmin}}
