\documentclass{article} % For LaTeX2e
\usepackage{nips14submit_e,times}
\usepackage{hyperref}
\usepackage{url}
\usepackage{amsmath,amssymb}
\usepackage{booktabs,multirow}
\usepackage[backend=bibtex,
            style=numeric,
            firstinits=true,
            natbib=true]{biblatex}
\usepackage{graphicx}
\usepackage{subfig}
\bibliography{nips2014lin}


\title{A Variational Framework for Gaussian Models with Nonlinear Likelihoods}


\author{
David S.~Hippocampus\thanks{Use footnote for providing further information
about author (webpage, alternative address)---\emph{not} for acknowledging
funding agencies.} \\
Department of Computer Science\\
Cranberry-Lemon University\\
Pittsburgh, PA 15213 \\
\texttt{hippo@cs.cranberry-lemon.edu} \\
\And
Coauthor \\
Affiliation \\
Address \\
\texttt{email} \\
}

% The \author macro works with any number of authors. There are two commands
% used to separate the names and addresses of multiple authors: \And and \AND.
%
% Using \And between authors leaves it to \LaTeX{} to determine where to break
% the lines. Using \AND forces a linebreak at that point. So, if \LaTeX{}
% puts 3 of 4 authors names on the first line, and the last on the second
% line, try using \AND instead of \And before the third author name.

\newcommand{\fix}{\marginpar{FIX}}
\newcommand{\new}{\marginpar{NEW}}
\input{notation}

%\nipsfinalcopy % Uncomment for camera-ready version

\begin{document}


\maketitle

\begin{abstract}
\end{abstract}

\section{Introduction}

TODO


\section{Nonlinear Gaussian Models}

In many inversion problems we may have a Gaussian prior model and a nonlinear
likelihood/forward model. We wish to find the latent input to the forward
model given its output by using Bayes' rule to perform the inversion,
\begin{equation}
    \probC{\lstate}{\obs} = \frac{\probC{\obs}{\lstate}\prob{\lstate}}
        {\prob{\obs}},
    \label{eq:bayesrule}
\end{equation}
where $\obs \in \real{d}$ is an observable quantity, and $\lstate \in \real{D}$
is the latent state of interest. The following forms are assumed for the prior
and likelihood:
\begin{equation}
    \prob{\lstate} = \gausC{\lstate}{\prmean, \prcov}
    \quad \text{and} \quad
    \probC{\obs}{\lstate} = \gausC{\obs}{\nonlin{\lstate}, \lcov},
    \label{eq:priorlike}
\end{equation}
where $\nonlin{\cdot} : \real{D} \to \real{d}$ is a nonlinear function or
forward model. Unfortunately the marginal likelihood is intractable to compute
because the nonlinear function makes the likelihood and prior non-conjugate,
\begin{equation}
    \prob{\obs} = \int \gausC{\obs}{\nonlin{\lstate}, \lcov}
        \gausC{\lstate}{\prmean, \prcov} d\lstate,
    \label{eq:marglike}
\end{equation}
which also makes the posterior intractable to evaluate. Thus, we choose to
approximate the posterior using variational inference \cite{}.


\section{Variational Approximation}

Using variational inference, we can put a lower bound on the log-marginal
likelihood using Jensen's inequality, 
\begin{equation}
    \log \prob{\obs} \geq \int \qrob{\lstate} \log 
        \frac{\prob{\obs, \lstate}}{\qrob{\lstate}} d\lstate,
\end{equation}
with equality iff $\KL{\qrob{\lstate}}{\probC{\lstate}{\obs}} = 0$, and
$\qrob{\lstate}$ is an approximation to the true posterior. This lower bound is
often referred to as `free energy', and can be re-written as follows
\begin{equation}
    \Fengy = \expec{q\lstate}{\log \probC{\obs}{\lstate}}
        + \expec{q\lstate}{\log \prob{\lstate}} + \entropy{\qrob{\lstate}},
    \label{eq:fengy}
\end{equation}
where $\expec{q\lstate}{\cdot}$ is an expectation with respect to the
variational posterior, $\qrob{\lstate}$. We assume the posterior takes a
Gaussian form,
\begin{equation}
    \qrob{\lstate} = \gausC{\lstate}{\pomean, \pocov}. \label{eq:varpost}
\end{equation}
Now we can evaluate the expectations and entropy terms in \eqref{eq:fengy},
\begin{align}
    \expec{q\lstate}{\log \probC{\obs}{\lstate}}
        =& - \frac{D}{2}\log{2\pi} - \frac{1}{2} \log\deter{\lcov} 
        - \frac{1}{2} 
            \expec{q\lstate}{\brac{\obs - \nonlin{\lstate}}\transpose\lcov\inv
            \brac{\obs - \nonlin{\lstate}}},
            \label{eq:qlike} \\
    \expec{q\lstate}{\log \prob{\lstate}}
        =& - \frac{D}{2}\log{2\pi} - \frac{1}{2}\log\deter{\prcov}
            - \frac{1}{2} \brac{\pomean - \prmean}\transpose\prcov\inv
            \brac{\pomean - \prmean}
            - \frac{1}{2} \trace{\prcov\inv\pocov},
            \label{eq:qprior} \\
    \entropy{\qrob{\lstate}} =& \frac{1}{2} \log\deter{\pocov} 
        + \frac{D}{2}\log{2\pi} + \frac{D}{2} \label{eq:qentropy}.
\end{align}
where the expectation involving $\nonlin{\cdot}$ may be intractable.


\subsection{Learning the Parameters}

TODO: Introduce the linearisation here, but I'm not sure if we also need the
augmented state form, it may be sufficient to say this can be shown to be
similar to a non-linear least squares problem in augmented form? Maybe put it
in for now, and take it out if we run out of space...

\subsection{Taylor Series Linearisation}

\subsection{Statistical Linearisation}

\subsection{Approximate Lower Bound}


\section{Linearised Gaussian Processes}

\section{Experiments}


\subsection{Various Nonlinear Functions}

Experimental setup:
\begin{itemize}

    \item True latent function, $\lstate_{\text{true}}\!\brac{\inobs} =
        \sin\!\brac{\inobs} + \cos\!\brac{\pi\inobs}$.

    \item The observations are generated as $\obs =
        \nonlin{\lstate_\text{true}} + \epsilon$ where $\epsilon \sim \gaus{0,
            0.2^2}$.
    
    \item Matern $\frac{5}{2}$ covariance functions are used. 

    \item Held out predictive negative log likelihood for the latent function
        is calculated as,
        \begin{equation}
            \frac{1}{2N\test} \sum^{N\test}_n \log{2\pi\pocovs\test_{nn}}
                + \frac{1}{\pocovs\test_{nn}} \brac{\lstates\test_n -
                    \pomeans\test_n}^2.
        \end{equation}

\end{itemize}

\begin{figure}[htb]
    \subfloat[][$\nonlin{\lstate} = 2\text{sign}\!\brac{\lstate} + \lstate^3$]
        {\includegraphics[width=0.5\linewidth]{fig/signdemo}\label{sub:sign}}
    \subfloat[][$\nonlin{\lstate} = \lfloor 5\lstate \rceil$]
        {\includegraphics[width=0.5\linewidth]{fig/rounddemo}\label{sub:rnd}}\\
    \subfloat[][MAP trace -- poor hyperparameter values]
        {\includegraphics[width=0.5\linewidth]{fig/trace_beg}\label{sub:mapb}}
    \subfloat[][MAP trace -- near optimal hyperprameter values]
        {\includegraphics[width=0.5\linewidth]{fig/trace_end}\label{sub:mape}}
    \caption[]{Example of learning the statistically linearised GP with
        non-differentiable nonlinear functions in \subref{sub:sign} and
        \subref{sub:rnd}. Typical traces from the MAP objective function used
        to learn $\pomean$, \subref{sub:mapb} is from near the outset of
        learning the hyperparameters, and \subref{sub:mape} is near model
        convergence -- both terminated because of `divergence'.}
    \label{fig:learnex}
\end{figure}

\begin{table}[htb]
    \centering
    \small
    \caption[]{Results from running nonlinear likelihood GPs with various
        differentiable nonlinear functions. The true latent function is
        $\lstate_{\text{true}}\!\brac{\inobs} = \sin\!\brac{\inobs} 
            + \cos\!\brac{\pi\inobs}$.}
    \begin{tabular}{r|c| c c c c c c}
        \multirow{2}{*}{$\nonlin{\lstate}$} & \multirow{2}{*}{Algorithm} & 
            \multicolumn{2}{c}{$-\frac{1}{N\test}
                \log\probC{\lstates\test}{\obs,\lstate}$} &
            \multicolumn{2}{c}{SMSE $\lstates\test$ (\expon{}{-4})} &
            \multicolumn{2}{c}{SMSE $\obss\test$ (\expon{}{-4})} \\
        & & mean & std. & mean & std. & mean & std.\\
        \toprule
        $\lstate$ 
            & Statistical & $-$1.4453 & 0.0601 & 36.471 & 7.309 & -- & -- \\
            & Taylor & 14.2072 & 2.7953 & 54.674 & 8.890 & -- & -- \\
            & \cite{Opper2009} \\
            & Linear & $-$1.2793 & 0.0162 & 40.210 & 7.012 & -- & -- \\
        \midrule
        $\lstate^3 + \lstate^2 + \lstate$ 
            & Statistical & $-$2.1402 & 0.0604 & 27.725 & 6.097 & 36.239 
                & 5.089 \\
            & Taylor & 10.1842 & 1.9443 & 31.094 & 9.193 & 36.991 & 5.869 \\
            & \cite{Opper2009} \\
        \midrule
        $\exp\!\brac{\lstate}$ 
            & Statistical & $-$1.3858 & 0.2320 & 99.405 & 75.756 & 170.07 
                & 27.490 \\
            & Taylor & 15.1875 & 4.0084 & 235.10 & 82.551 & 174.70 & 26.181 \\
            & \cite{Opper2009} \\
        \midrule
        $\sin\!\brac{\lstate}$ 
            & Statistical & $-$0.8892 & 0.1259 & 152.75 & 32.744 & 858.63
                & 104.26 \\
            & Taylor & 20.1577 & 5.9408 & 387.67 & 131.13 & 889.80 & 127.22 \\
            & \cite{Opper2009} \\
        \midrule
        $\tanh\!\brac{2\lstate}$
            & Statistical & $-$0.5903 & 0.1132 & 395.96 & 40.270 & 602.24 
                & 60.622 \\
            & Taylor & 23.7728 & 8.9093 & 894.32 & 523.64 & 606.16 & 64.206 \\
            & \cite{Opper2009} \\
        \bottomrule
    \end{tabular}
\end{table}


\subsection{Binary Handwritten Digit Classification}

We use the error rate (percent incorrectly predicted labels) and average 
negative log Bernoulli likelihood to asses performance,
\begin{equation}
    - \frac{1}{N\test} \sum^{N\test}_{n} \obss\test_n 
        \log\brac{\pi\test_n}
    + \brac{1-\obss\test_n}\log\brac{1 - \pi\test_n},
\end{equation}
where $\pi\test_n = \expec{}{\prob{\obss\test_n = 1}}$ is predicted from the
classifications algorithms.

A square exponential kernel with amplitude $\sigma_\text{se}$ and length
scale $l_\text{se}$ is used for the Gaussian processes in this experiment.

\begin{table}[htb]
    \centering
    \small
    \caption[]{Binary Gaussian process classifier performance on the USPS
        handwritten-digits dataset for numbers `3' and `5'.}
    \begin{tabular}{r| c c c c}
        Algorithm & Av.\ neg.\ ll. & Error rate (\%) 
            & $\log\!\brac{\sigma_\text{se}}$ & $\log\!\brac{l_\text{se}}$ \\
        \toprule
        GP -- Laplace & 0.11528 & 2.9754 & 2.5855 & 2.5823 \\
        GP -- EP & 0.07522 & 2.4580 & 5.2209 & 2.5315 \\
        SVM (RBF) & 0.08055 & 2.3286 & -- & -- \\
        Logistic Reg. & 0.11995 & 3.6223 & -- & -- \\
        \midrule
        GP -- Statistical & 0.11748 & 2.9754 &  2.8177 & 2.5504 \\
        GP --Taylor & 0.69293 & 42.4321 & 3.1268 & $-$0.1696 \\
        \bottomrule
    \end{tabular}
\end{table}

\section{Conclusion}

\subsubsection*{Acknowledgments}

\subsubsection*{References}
\printbibliography

\end{document}
